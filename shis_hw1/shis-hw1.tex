% Latex2e homework template for 14-741/18-631 taught by Nicolas Christin
%
\documentclass[12pt]{article}  % Decrease to 10pt if desired

\usepackage[letterpaper,margin=1in]{geometry}
\usepackage{times}
\usepackage{ragged2e}
\usepackage{lastpage}
\usepackage{fancyhdr}
\usepackage{url}
\usepackage{mathtools}
\pagestyle{fancy}
\addtolength{\parskip}{\baselineskip} %--skip lines between paragraphs
\setlength{\parindent}{0pt} %--don't indent paragraphs
\AtBeginDocument{\raggedright}

%
% Set your personal information here
%
\newcommand{\FirstName}{Shi}
\newcommand{\LastName}{Su}
\newcommand{\AndrewID}{shis}
\newcommand{\CourseNumber}{14-741}
\newcommand{\Campus}{PIT}  % Either PIT or Kobe

%
% Set assignment date here
%
\newcommand{\Date}{10/10/2015}


% Set up headers
\renewcommand{\headrulewidth}{0pt}
\setlength{\headheight}{0.5in}
\lhead{\FirstName~\LastName\\\AndrewID}
\chead{}
\rhead{\CourseNumber~/~\Campus\\\Date}
\lfoot{}
\cfoot{}
\rfoot{Page \thepage~of \pageref{LastPage}}

%Answer block
\newenvironment{answer}{
\setlength\parindent{24pt}
}

%
% Begin document
%
\begin{document}

%
% Problem 1
%
\begin{center}
\textbf{Problem \#1 : Using PGP}
\end{center}
\parskip 0pt
Email: shis@andrew.cmu.edu \\
OpenPGP keyID: 58053E5D \\
Fingerprint: 2BA3 6514 F478 CF9F 5C06  0C04 A6CB 1835 5805 3E5D \\
Public key is available in {\bf shis\_hw1\slash 58053E5D.asc} \\
\medskip
1. Based on this information, how do you verify that the public key you got from the web page is valid, i.e., that no one has modified it?

\medskip
\begin{answer}
By comparing the downloaded key's fingerprint with the supposedly correct fingerprint provided in the assignment handout, it there're the same, the key is authentic. Because the fingerprint is a hash of the user's certificate [1], if the key get modified, the hash (fingerprint) will change as well, and the course handout downloaded is from a trusted resources, so we can assume the fingerprint is correct.
\end{answer}
\bigskip
3. If you had to send Mahmood an attachment, would you use PGP/MIME or PGP/inline, knowing that the various email client(s) he uses support both methods? Justify your answer.

\medskip
\begin{answer}
PGP/MIME, because PGP/inline only encrypt the plain text inside the email message, while PGP/MIME allows you to encrypt attachment together with the message body. [2]
\end{answer}
\bigskip
[1] Pgpi.org,. (2015). How PGP works. Retrieved 5 October 2015, from http://www.pgpi.org/doc/pgpintro/\#p11

\smallskip
[2] Enigmail.net,. (2015). Enigmail: Enigmail Configuration Manual. Retrieved 5 October 2015, from https://www.enigmail.net/documentation/per-account.php


%
% Problem 2
%
\newpage
\begin{center}
\textbf{Problem \#2 : Padding attack}
\end{center}

Python script available in {\bf shis\_hw1\slash padding\_attack\slash}

\medskip
{\bf Decryption Function}\\

\medskip
shis\_decticket.py

\medskip
When running without arguments, the script decrypt the cipher provided in the
homework.

Use ``-c" to provide a different cipher\\
Use ``-h" for more information\\
\bigskip
{\bf Encryption Function}\\
\medskip
shis\_goldticket.py\\
\medskip
This python script is importing function from shis\_decticket.py,
so please keep them under same directory.\\
\medskip
When running without arguments, the script encrypt the following ticket
\{``username":``shis",``is\_admin":``true",``expired":``2020-01-31"\}\\
\medskip
Use ``-u" to provide a different username, e.g. ``instructor"\\
Use ``-a" to provide admin setting, ``true" or ``false"\\
Use ``-e" to provide expiration date, e.g. ``2016-01-31"\\
Use ``-h" for more information\\

%
% Problem 3
%
\newpage
\begin{center}
\textbf{Problem \#3 : Finding MD5 collisions}
\end{center}

1. A succinct description of how in practice the attack works. Do not delve into the cryptographic details, simply assume the result of Wang and Yu is well known
\begin{enumerate}
  \setlength{\itemsep}{1pt}
  \setlength{\parskip}{0pt}
  \setlength{\parsep}{0pt}
\item Write a script file contains 2 methods perform different tasks, compile it with a main body contains a switch to choose between the methods using 2 placeholder ``crib" strings.\\
\item Read the first block and get the MD5 initial vector of the executable generated from step 1.\\
\item Given the initial vector, find the MD5 collision pair using the method of Wang and Yu.\\
\item Copy the executable file from step 1 to create 2 executable, at the same time replace the ``crib" string with one of the collision pair in each new executable. 
\end{enumerate}
\begin{answer}
Thus most of the content of the 2 executables are the same, except the collision pair. Because of the switch in the main function, they have different behavior, and according to Wang and Yu's work, they will have same MD5 hash.
\end{answer}
\bigskip
2. Generate MD5 collision executables\\
\begin{answer}
Executables available in {\bf shis\_hw1\slash MD5\_collisions\slash}
\end{answer}
\bigskip
3. Answers to the following questions:
\begin{itemize}
      \item What is the linearity property that makes the attack work for files of arbitrary length?

      According to Wang and Yu's attack, giving arbitrary IV, it's able to find two pairs of blocks M,M' and N,N', which makes f(f(s, M), M') = f(f(s, N), N'), f is the hash function applied to each block. Because of the function f's input is only the output of previous block, this linearity property ensure that after MD5 function processed these 2 blocks, it generates same output for next block. Thus, as long as the blocks before/after M,M' and N,N' are the same, they can have arbitrary length.
          
      \item We saw here how to use this attack on an executable. How would you go about implementing such attack on a document? Which feature must the document format have? Is it possible to carry out such an attack in a completely stealth manner on an ASCII document?

     Implementing this attack on document is basically the same as on executable file, put 2 version of content in one document, use generated MD5 collision block pair to control which version to display. This method from  Selinger requires each file contains both version of content, so the document format must support hidden contents and constructing dynamic content, for example pdf and tiff.[1] So it's not possible carry this attack on an ASCII document. In an ASCII document, there's no way to hide the extra data especially the ascii characters.
\end{itemize}

[1] Csrc.nist.gov,. (2015). Retrieved 12 October 2015, from http://csrc.nist.gov/groups/ST/hash/documents/Illies\_NIST\_05.pdf

%
% Problem 4
%
\newpage
\begin{center}
\textbf{Problem \#4 : Public Key Cryptography}
\end{center}

1. Calculate the private key d:

When knowing p = 23, q = 17, e = 3, we can get: N = pq = 23 * 17 = 391\\
And d is the modular multiplicative inverse of e, from {\it ed = 1 (mod (p-1)(q-1))}:\\
{
\setlength\parindent{24pt}
ed - k(p-1)(q-1) = 1\\
3d - 1 = 352k\\
3d - 352k = 1\\
}
Use Euclid's Algorithm to compute gcd(352,3)\\
{\setlength\parindent{24pt}
352 = 117 * 3 + 1\\
3 = 3 * 1 + 0\\
}
Then we get: 1 = (-117)*3 + 352, to get a positive d, add 352*3 - 3*352 into equation\\
{
\setlength\parindent{24pt}
1 = (-117 )*3 + 352*3 - 3*352 +352\\
1 = 235*3 - 2*352\\
}
So private key {\bf d = 235}\\
\medskip
2. Describe which party (keyboard/SecApp) knows which keys (public key/private key), and show the steps of encryption and decryption of the exemplary keyboard input `B'.\\
\medskip
\begin{answer}
     Keyboard knows the public key in order to encrypt the input, only SecApp knows the private key, so only it can decrypt the message. 
     When sending 'B' (ascii: 66), keyboard encrypted it with public key (n,e)=(391,3), according to {\it \(m^e = c\)(mod n)}:\\
\end{answer}
     {\setlength\parindent{24pt}
          \(66^3 = c (mod 391)\), c = 111, so `B' is encrypted to 111\\
     }
     When SecApp decrypting the message with private key d, according to {\it \(c^d = m (mod n)\)}:\\
          {\setlength\parindent{24pt}
          \(111^{235}= m (mod 391)\), so the app can calculate m = 66.\\
          }
\medskip          
3. List one advantage and one disadvantage for picking e=3 as the public key?\\
\medskip
\begin{answer}
     Use e=3 such a small exponent as the public means less calculation which can make the encryption or signature verification process faster. However the small exponent may also be less secure in some cases, for example when the message is not padded simply or the private key is partially exposed, etc. [1]
\end{answer}
\medskip
4. Another company needs your technical input to beat this product. Please list 3 different arguments for why you think this design is not good.
\begin{enumerate}
  \setlength{\itemsep}{1pt}
  \setlength{\parskip}{0pt}
  \setlength{\parsep}{0pt}
\item p,q are too small, which results in a small n=pq, making the factorization of it much easier. When p,q got revealed, it's simple to calculator private key d with {\it ed mod(q-1)(p-1)=1}
\item Because of the information is sending from keyboard, suppose it sends each key typed right away, that means for each encryption the length of plain text is 1 and is likely subject to short plain text attack. Also it only has limited variation of inputs, so if the algorithm doesn't have a good padding scheme, an attacker may collect large amount of cipher texts and apply frequency analysis attack on them. If the keyboard encrypts bunch of keys together, it can lead to input latency.
\item Asymmetric encryption method like RSA is generally much slower to calculator than those symmetric ones, and keyboard is a real time input device, the delay for encrypting and decrypting may largely sacrifice the user experience.
\end{enumerate}

[1] Boneh, D. (1999). Twenty years of attacks on the RSA cryptosystem. Notices of the AMS, 46(2), 203-213.

\end{document}