% Latex2e homework template for 14-741/18-631 taught by Nicolas Christin
%
\documentclass[12pt]{article}  % Decrease to 10pt if desired

\usepackage[letterpaper,margin=1in]{geometry}
\usepackage{times}
\usepackage{ragged2e}
\usepackage{lastpage}
\usepackage{fancyhdr}
\usepackage{url}
\usepackage{mathtools}
\usepackage{listings}
\usepackage{graphicx}
\pagestyle{fancy}
\addtolength{\parskip}{\baselineskip} %--skip lines between paragraphs
\setlength{\parindent}{0pt} %--don't indent paragraphs
\AtBeginDocument{\raggedright}

%
% Set your personal information here
%
\newcommand{\FirstName}{Shi}
\newcommand{\LastName}{Su}
\newcommand{\AndrewID}{shis}
\newcommand{\CourseNumber}{14-741}
\newcommand{\Campus}{PIT}  % Either PIT or Kobe

%
% Set assignment date here
%
\newcommand{\Date}{11/20/2015}


% Set up headers
\renewcommand{\headrulewidth}{0pt}
\setlength{\headheight}{0.5in}
\lhead{\FirstName~\LastName\\\AndrewID}
\chead{}
\rhead{\CourseNumber~/~\Campus\\\Date}
\lfoot{}
\cfoot{}
\rfoot{Page \thepage~of \pageref{LastPage}}

%Answer block
\newenvironment{answer}{
\setlength\parindent{24pt}
%\par\addvspace{\baselineskip}
}


%
% Begin document
%
\begin{document}

%
% Problem 1
%
\begin{center}
\textbf{Problem \#1 : Running Kerberos and Wireshark}
\end{center}
\parskip 0pt

\begin{figure}[h]
\centering
  \includegraphics[scale=0.8]{duang.jpeg}\\
 \caption{Modules of Gravel}
 \end{figure}
 
1. Why is the last step (Step 6) optional?\\
\begin{answer}
aaa
\end{answer}
\medskip

2. kinit\\
(a) Which version of Kerberos are you running? \\
\begin{answer}
aaaa
\end{answer}
\medskip

(b) What is the server name? (AS))\\
\medskip
\begin{answer}
The buffer fills used in the exploit:\\
\end{answer}
\medskip

(c) What is the client name? (C)\\
\begin{answer}
aaaaa
\end{answer}
\medskip

(d) What encryption method is being used?\\
\begin{answer}
aaaaa
\end{answer}
\medskip

(e) What is the encrypted value of the ticket TicketTGS in the lecture notes (use only the first 8 hexadecimal digits)?\\
\medskip

3. Provide the output of klist. How many tickets are currently valid? For how long are they valid? Who are the principals involved?\\

\begin{answer}
aaaaa
\end{answer}
\medskip

4. AFS\\
(a) Show the four Kerberos messages that preside over the establishment of the AFS connection.\\
\begin{answer}
aaaaa
\end{answer}
\medskip

(b) What is the name of the AFS server? (V )\\
\begin{answer}
aaaaa
\end{answer}
\medskip

(c) Show that the ticket(TicketTGS) the authentication server gave you is sent to the ticket granting server.\\
\begin{answer}
aaaaa
\end{answer}
\medskip

(d) Identify the ticket that the TGS returned to you (T icketV ), and show that it is sent to the AFS server when you are trying to create the file 14741-test. Show only the first eight hexadecimal digits of the ticket.\\
\begin{answer}
aaaaa
\end{answer}
\medskip

5. Once again, provide the output of klist. How many tickets are currently valid? For how long are they valid? Who are the principals involved?\\
\medskip

\begin{answer}
aaa
\end{answer}


%
% Problem 2
%
\newpage
\begin{center}
\textbf{Problem \#2 : TPM, PHP, and HTTP}
\end{center}

1. Explain why this reasoning is completely flawed.\\
\begin{answer}
aaaaaa
\end{answer}
\bigskip

2. Whether or not the TPM could prevent the attacks from succeeding. If the attacks can be foiled, explain how. If they cannot, state why:\\
\begin{answer}
aaaaaa

\end{answer}
\medskip

3. Is the update procedure secure? Justify your answer, by either proving its security, or giving an example of attack against it.\\
\begin{answer}
aaaaaa

\end{answer}
\medskip

%
% Problem 3
%
\newpage
\begin{center}
\textbf{Problem \#3 : Alice and Bob getting married}
\end{center}

1. Which security property/ies Bob?s protocol enforces?\\
\begin{answer}
aaaaaa

\end{answer}
\medskip

2. Show that Alice?s father is wrong ? in that one of the security properties Bob?s protocol enforces is not maintained anymore.\\
\begin{answer}
aaaaaa

\end{answer}
\medskip

3. Bob?s protocol unfortunately has a major problem: It is vulnerable to a replay attack in case the same message X is repeated over time. Enhance the protocol proposed by Bob to prevent this attack.\\
\begin{answer}
aaaaaa

\end{answer}
\medskip

4. Enhance the protocol proposed by Bob to provide the freshness property.\\
\begin{answer}
aaaaaa

\end{answer}
\medskip

%
% Problem 4
%
\newpage
\begin{center}
\textbf{Problem \#4 : Finding open ports and bypassing firewalls}
\end{center}

1. Suggest a technique to exhaustively determine all the open TCP ports on a given host you want to attack.\\
\begin{answer}
aaaaaa

\end{answer}
\medskip

2. Harry Bovik claims the attack consumes a lot of memory on the attacker?s side. Is he right or not? Why?\\
\begin{answer}
aaaaaa

\end{answer}
\medskip

3. Suggest an alternative method to determine all the open TCP ports on the host you want to attack.\\
\begin{answer}
aaaaaa

\end{answer}
\medskip

4. Harry Bovik tells you that neither of these attacks work to determine all open ports on a packet filtering layer-3 firewall. Why is he right?\\
\begin{answer}
aaaaaa

\end{answer}
\medskip

5. Suggest an extension to either of the above attacks that allows to figure out open ports on a firewall.\\
\begin{answer}
aaaaaa

\end{answer}
\medskip

\end{document}